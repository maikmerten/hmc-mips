% PROJECT PROPOSAL TEMPLATE
%
% This document was prepared for use with pdflatex.

% Lines beginning with a % are comments.
\documentclass[a4paper,12pt]{article}

% The amsfonts package contains some useful mathematical symbols
\usepackage{amsfonts}

% The graphicx package allows you to import JPG and PDF images
\usepackage[pdftex]{graphicx}

% The enumerate pachage permits fancy enumerated lists
\usepackage{enumerate}

% Setup the page
\topmargin -1.5cm
\textheight 25.0cm
\oddsidemargin -0.0cm
\textwidth 16.5cm
\pagestyle{plain}
\setcounter{page}{0}
\linespread{1.3}

\begin{document}

%*******************************************************************
% Draw the title page
% The only things you are likely to need to edit are marked with
% a **** comment ****
\begin{titlepage}
\vspace{-1.5cm}
\begin{center}
\includegraphics[width=2cm]{ualogo_colour.jpg}
\vspace{1cm}

\textbf{\large THE UNIVERSITY OF ADELAIDE}\\

\textbf{SCHOOL OF ELECTRICAL \& ELECTRONIC ENGINEERING}

{\small ADELAIDE, SOUTH AUSTRALIA, 5005}

\vspace{1.5cm}
\textsc{Implementation Plan}
\vspace{1cm}

% **** Insert your project title here ****
\textbf{\LARGE MIPS Microprocessor\\(Cache Circuits)}

\vspace{1cm}
% **** Insert the group members' names here ****
%{\Large Group Members}
\textbf{Joel Stanley, Rhys Bowden, Melanie Say Wei Tan, Robert Moric}

\vspace{3.5cm}
\textbf{\large ELEC ENG 4039 A/B HONOURS PROJECT}

\vspace{1ex}
\setlength{\linespread}{1}
\textbf{B.E. in Electrical and Electronic Engineering\\
B.E. in Computer Systems Engineering\\
B.E. in Telecommunications Engineering\\}
\end{center}

\vfill
Each student at Level IV of the course in Electrical and
Electronic Engineering, Computer Systems Engineering and
Telecommunications Engineering is required to complete this course.
The course involves approximately 240 hours of project work over the
whole academic year.  Students are assessed on their performance in
the project, a written proposal, this written report, a technical
paper, and two seminar presentations.

\vspace{1cm}
Date submitted:

\vspace{1ex}
Supervisor:

\vspace{1ex}
Signature of Supervisor:
\end{titlepage}
% End of the title page
%*******************************************************************

\newpage
% The \section* command creates a section without a number
\section*{Objectives}
The MIPS architecture is popular Reduced Instruction Set Computer (RISC) microarchitecture, utilised in many commodity hardware sytems.
\\In collaboration with Dr David Money Harris' VLSI Design class from Harvey Mudd College, California, the project involves implementation of the MIPS R2000-class microprocessor to be fabricated by MOSIS. Our team will focus on the cache memory system, followed by projects utilising the fabricated hardware.
\\The testbed containing the microprocessor and supporting perephials will be packaged in a form that will enable it to be used as a promotional tool, as well as for demonstrating the unit.

\newpage
\tableofcontents

\newpage
% The \section command creates a numbered section
% Note that there are also \subsection and \subsubsection commands.
\section{Background and Significance}

The first aim of this project is to co-operate with the VLSI class
from Harvey-Mudd College, California, to design and build a MIPS-based
microprocessor.

Microprocessors are electronic devices that contain all the functions
of a CPU on a single integrated circuit. This microprocessor will be
split into a control unit, a coprocessor and a data path. The datapath
contains a fetch, decode, execute and memory stage. The Adelaide team
is designing the 512kB cache for the memory stage. A cache stores
recently used memory data on chip to save slow memory accesses. The
cache is direct mapped, meaning that each datum from memory has a
unique place it can be stored in the cache.

The chip will use a MIPS R2000 instruction set architecture
(ISA). MIPS stands for Microprocessor without Interlocked Pipeline
Stages. The MIPS ISA is a popular RISC microprocessor
architecture. R2000 has 32-bit instructions including various loads,
stores, arithmetic, jumps and branches, shifts, moves and exceptions.

In the production of a microprocessor testing is required on both the
design and the finished hardware.

\section{Project Specifications}

\subsection{Final Deliverables}
\begin{enumerate}[{[D}1{]}]
\item Hardware Presentation of the MIPS-based microprocessor
\item Package of testing tools and report on test results
\item Extension: Report examining low power design alternatives
\end{enumerate}

\subsection{Requirements}
\begin{enumerate}[{[R}1{]}]
\item Hardware demonstration
  \begin{itemize}
  \item Running an interactive program (possibly a Web-server or
    ELIZA).
  \item Robust packaging so that the demonstration is easy to
    move and set up.  
  \item Uses MIPS-based microprocessor.
  \end{itemize}

\item Testing tools
  \begin{itemize}
    \item Software tests of design using either fast SPICE or IRSIM,
      depending on availability.
    \item Documentation for software testing.
    \item Detailed methods of hardware testing.
  \end{itemize}
\item Extension: Report on low power
  \begin{itemize}
  \item Examination of MIPS based microprocessor.
  \item Development of alternative designs to achieve low power.
  \item Examination and evaluation of alternative designs
  \end{itemize}
\end{enumerate}

\subsection{Reporting Requirements}

First Semester
\begin{description}
\item[Week 4 Project Implementation Plan:] A plan detailing what is to
  be accomplished during the project, and how the group intends to do
  it.
\item[Week 5 Proposal Seminar:] Presents and explains the intended
  project including requirements and what will be produced.
\item[Week 8 Critical Design Review:] Evaluate and assess intended
  design(s) for project.
\item[Week 10 Submit Peer Review:] Review another group's project.
\item[Week 12 Project Log Book:] A record of all meetings and time
  spent on the project.
\end{description}

Second Semester
\begin{description}
\item[Week 9 Final Project Report and Technical Paper due:] The final
report covering the completed project and the process of producing it.
\item[Week 10 Final Project Seminars:] A seminar on the completed
project.  
\item[Week 11 Project Exhibition:] Displaying the project to the
public.
\end{description}

%\subsection{Acceptance Criteria}
%
%State how the customer will measure whether the project has been a
%success.

%ROBS SECTION

\section{Proposed Approach}

\subsection{System Architecture}

The cache is the focus of our project. A thorough description of
the memory system for this project can be found in
\cite{hmcmemsys}. The MIPS microprocessor has been designed by the
microarchitecture team from HMC under the supervision of Professor David
Harris. An interface of the entire system has been defined and all modules used to perform
the logical functions have been described in Verilog using a development tool
called ModelSim. Testing the completed Verilog system demonstrated the logical
functionality to be correct. Translation of the Verilog code into schematics
has been completed. Schematics are currently in the process of being
tested and refined using a VLSI design application called Electric8.04. The
remainder of the design process will be elaborated in Section 3.2.

Figure~\ref{cacheblock} is the current block diagram of
the cache system. Major modules are described below:
\begin{enumerate}
\item \textbf{Cache Controller: }As its name suggests this module provides the logic to generate control signals to govern the read/write behaviour of the cache and to generate status signals to inform the memory system of the current state of the cache.
\item \textbf{Cache RAM: }In the overall memory system there are two identical caches and are both synchronous with the microprocessor. These are Data and Instruction caches. They are each 512 bytes capacity. Each cache can hold 128 words given that a word is 4 bytes. In a cache each word has an associated tag. In order to reference each word, 7 bits would be required resulting in the tag being 7 bits long. The remaining bits of the word must hold the data. Given that each word is 32 bits long with 7 bits used for the tag and the requirement that all addresses be word aligned we would therefore have  32 - 7 - 2 = 23 bits of tag data. Additionally, a valid bit must be associated with each tag. The total width of each cache slot is then 23 + 1 + 32 = 56 bits. The total size of the cache's memory is then 128*56 = 7168 bits or 896 bytes per cache.
\item \textbf{Data Output: }This module implements the multiplexing functionality to manage a bi-directional data line for reads and writes of the cacheram. Since the external memory only has one data and address bus, the memory system is responsible for multiplexing read/writes to the external memory from the two caches. The Adelaide team will not be involved in design of this module.
\end{enumerate}

\begin{figure}
\centering 
%\includegraphics[width=\textwidth]{cacheblock}
\includegraphics[width=\textwidth]{cacheblock.jpg}
\caption{Block diagram of the MIPS cache.}
\label{cacheblock}
\end{figure}

The Adelaide team will be involved in the continued design and
verification of the Cache Controller and the Cache Ram. \cite{hmcmemsys} details the functions performed by the cache. A summary of the features is given below:
\begin{enumerate}
\item \textbf{Write-through: }When a data write is requested from the processor, it is written immediately to memory (or write buffer) as well as the cached. The cache never holds a newer copy of data than main memory (or write buffer).
\item \textbf{Write buffer: }To improve performance, so the CPU does not stall for the entire external memory write time, a FIFO (first-in, first-out) write buffer is used. Once the write buffer is full, the CPU stalls until a space is available in the write buffer.
\item \textbf{Direct-mapped: }The lower seven bits of the memory address are used as the tag in the cache memory.
\item \textbf{Physically addressed: }The address the cache uses in the tag data is based upon the physical address of the data in external RAM.
\item \textbf{Bypassing: }The cache can be bypassed via the upper bits (explained in the memory map section) so that certain data (e.g. memory mapped I/O) is never cached.
\item \textbf{Swapping: }The caches can be swapped. This is mainly useful for cache invalidation during boot loading of the processor. 
\end{enumerate}

The modules used in the cache are described in further detail in Figure~\ref{cachecontroller} and
Figure~\ref{cacheram} respectively.

\subsection{Cache Controller}

This module comprises the following components: 
\begin{enumerate}
\item \textbf{Address Tag Data Logic: }This module is used to generate the 'bypass' and 'done' cache control signals. Bypass control signal is generated upon detection of data to be bypassed. The cache bypass signal prevents the cache memory from performing a data caching operation on the data. The Done control signal is used to indicate the completion of a read or write operation.
\item \textbf{Controller State Logic: }This module is used to generate the 'waiting' and 'reading' control signals.
\end{enumerate}

\subsection{Cache Ram}

This module contains the following components:
\begin{enumerate}
\item \textbf{64-bit Decoder: }Inputs to the decoder are six address lines. These inputs are then processed by the decoder logic to determine which of the 64 bit wordlines of the SRAM array to excite. Once a wordline is excited, it is possible to read from or write to the location specified by that address.
\item \textbf{Cache Signal Buffers: }This section of logic does not perform any specific function at RTL. It is simply used to drive large capacitance loads that appear on the signal line outputs due to the large fanout from the cache ram array. Testing with Verilog does not require this module as Verilog does not consider path delays or driving power of a signal. It only considers logical behaviour. This module will be retained for Verilog tests to ensure that no logical errors were introduced in the design of the buffers.
\item \textbf{SRAM Array: }This module is a 64x53 matrix of 6T SRAM cells. Each cell is used for storage of a single 'bit' of data. It is a very common design used in cache rams because of its speed, reliability and dense layout.
\item \textbf{Bitline Conditioning: }Bitlines are a part of the SRAM cell design. When the SRAM array is all connected the bitlines appear as a pair of long wires running down a single bit column of each wordline. The bitlines are used to read data from the cell by being pre-charged in sync with a clock signal and then driven high or low by the data value of the SRAM cell. Writes are accomplished through the use of a write driver as described below.
\item \textbf{Write Driver: }The role of this module is to condition the bitlines to contain the desired value to be stored in the SRAM cell. Once the bitlines are conditioned and the control signals for a write have occurred, the voltage on the bitlines will force the SRAM to take the values on the bitline thereby writing their value into memory.
\end{enumerate}

\begin{figure}
\centering 
\includegraphics[width=\textwidth]{cachecontroller.jpg}
\caption{The MIPS cache controller module.}
\label{cachecontroller}
\end{figure}

\begin{figure}
\centering 
\includegraphics[width=\textwidth]{cacheram.jpg}
\caption{The MIPS cache ram module.}
\label{cacheram}
\end{figure}

\subsection{Design Processes}

Figure~\ref{designflowALL} describes the methodology that will be followed to complete the stages of development of the cache system, project extensions and, public presentation. Initial development and testing of the cache memory system is done in conjunction with the HMC team. The remaining stages are done separate from the HMC team and involves evaluating the temporal performance of the design, the evaluation of the fabricated microprocessor, packaging, and setting up a software environment on the packaged system. The first stage begins with the Verilog specification that has already been developed and verified. This starting point involves learning Verilog, a hardware description language, and then determining what modules were used and the overall interface for the memory system. The design of the system in Verilog has been developed using ModelSim. Familiarity with this development environment is therefore a requirement.

The next stage of the design flow involves designing schematics from the Verilog system description. This is possible since this Verilog source code contains the RTL description for each module. Schematics will be built using the VLSI design suite Electric. Once the schematics are designed a Verilog deck containing the schematic netlist of a module will be extracted for use in Verilog testbenches in ModelSim. This will enable incremental development and checking of individual schematics for logical correctness. Care needs to be taken at this stage to match the interface of the schematic with the rest of the system. As an alternative to Verilog test benches for smaller leaf cells, a faster option is to run IRSIM within Electric and apply a short series of stimuli to check on logical behaviour.

Once the schematics have been designed and verified, layouts of the schematics will commence. A similar testing procedure will be conducted to the schematics for the layouts in addition to the MOSIS design rules for layout, Schematic vs. Layout verification features built-in to Electric.

At present, further testing procedures planned involve examining the timing aspects of the circuitry giving a more realistic assessment of the design performance. In order to carry out such testing it is necessary to generate testbenches for simulation tools that have the capacity to measure temporal behaviour such as IRSIM or a SPICE based application. Testbenches for IRSIM or SPICE may be generated using the existing Verilog source code. There are a number of possible ways of doing this. Firstly, it may be possible to insert trace statements within the Verilog source code to generate the stimulus for a command file for IRSIM or SPICE. Secondly, it may be possible to parse the Verilog command dump files (.vcd files) and convert it into a form useable by IRSIM. A SPICE package such as Fast Spice, HSPICE or Nano Spice would be preferable over IRSIM due to its better accuracy. As an alternative to the above testing process, carrying out the entire testing procedure using a professional software package such as Cadence or Synopsis would be the most desirable due to the high quality of the simulation engine and built in testing features. Should these tools be unavailable or too difficult to configure, IRSIM will be the default testing tool.

As testing of circuits progresses circuits that are found to have poor temporal performance that may violate timing requirements will be examined and redesigned at the schematic level. The layouts of these sections will then be redone to reflect the changes at the schematic level. Once the design is found to be free from timing problems the design may be sent off for fabrication.

After the design is sent for fabrication, various extensions to the project are to be implemented in parallel. These include design modifications to introduce low power design to the microprocessor, testing of the fabricated microprocessor, installing a software package on the packaged system, and developing a packaging for the microprocessor for public presentation including peripherals.

Low power design will focus on the most significant source of power consumption in a microprocessor, namely dynamic power dissipation. The principles that will be followed for reduction of dynamic power dissipation will include optimising some circuitry to inactivate unused blocks, and improving the defined architecture of the implementation of the MIPS microprocessor to achieve the same functionality with less processing. The designs produced will be implemented and evaluated.
%joel's bit
The hardware testing of the processor is split into two sections, being firstly simulated hardware in the form of Verilog implemented on FPGA, and once it is available, fabricated hardware.

The initial hardware testing will consist of two FPGA boards, one containing a memory input output controller and associated RAM for the CPU to operate, the other containing a FPGA implementation of our CPU.

Once the fabricated hardware becomes available, the FPGA board containing the simulated CPU will be replaced with a PCB, designed by the HMC team. It will provide a hardware socket for the microprocessor package, associated power circuitry, and a LCD for displaying outputs. It will use the FPGA memory for it's RAM needs.

The software will be programmed directly into the memory on the FPGA. A bootloader will be developed, along with a C compiler toolchain to enable simple software to be run on the board.

As an extension, it will be attempted to build a GNU toolchain, comprising of binutils, gcc and either ulibc or libc. This toolchain will then be used to compile GNU Linux to run on the board.

The system will be both a testbed and demonstration unit, to be put on display for the public to view the operation of the system.

\begin{figure}
\centering 
\includegraphics[width=\textwidth]{designflowALL.jpg}
\caption{Flow chart for the design process for the project.}
\label{designflowALL}
\end{figure}



%END ROBS SECTION


%MELS SECTION

\section{Project Plan}

\subsection{Interim Deliverables}
\begin{enumerate}[{[ID}1{]}]
\item Cache designs for MIPS-based microprocessor
\item MIPS microprocessor in package
\item Software for MIPS
\item Documentation: 
  \begin{itemize}
  \item Chip report (HMC requirements)
  \item Reporting requirements
  \end{itemize}
\end{enumerate}


%\subsection{Milestones}
% The \label command marks a section (or a table, equation or figure) so 
% that it can be referenced elsewhere in the text.
%\label{sec:milestones}
%\begin{description}
%\item[M1 01/01/07:] list the key milestones for your project.
%\item[M2 01/01/07:] each final deliverable will have a deadline. These
%  should be milestones.
%\item[M3 01/01/07:] assign deadlines to the most important interim
%  deliverables. These should also be milestones.
%\end{description}

\subsection{Work Breakdown}

This section is to be read in association with the accompanying Gantt Chart.

\begin{enumerate}
\item \textbf{Project Planning:} (26th Feb 2007 - 2nd March 2007) 

\item \textbf{Devise Test Plan:} (3rd March 2007 - 5th March 2007)

\item \textbf{Block Schematic Simulate:} (milestone: 5th March 2007)

\item \textbf{Project Implementation Plan:} (milestone: 22nd March 2007)

\item \textbf{Block Level Design Validation:} (6th March 2007 - 25th March 2007)

\item \textbf{Proposal Seminar:} (milestone: 26th March 2007)

\item \textbf{System Level Design Validation:} (27th March 2007 � 29th April 2007) 
  \begin{enumerate}[a)]
    \item EDA tools (Rob)
    \item VCD parse (Joel)
    \item Verilog trace writes (Mel) 
    \item Snoopgen (Rys)
    \item Testing (All)
  \end{enumerate}

Tasks a) to d) can be done in parallel to each other by assigned group members. Testing will commence after the implementation of each task.

\item \textbf{Critical Design Review:} (milestone: 30th April 2007)

\item \textbf{Project Extension Planning:} (1st May 2007 � 10th May 2007)

\item \textbf{Extensions:} (18th May 2007 � 13th August 2007)
The project extension can be divided into 4 different parts:
  \begin{enumerate}[a)]
    \item Low Power Design (Rob and Rys)
    \item Interactive Demonstration Program for the chip (Joel)
    \item Peripherals for the chip (Rys) 
    \item Packaging design (Mel)
  \end{enumerate}

Research and implementation of the four different parts will commence in mid of May. Each part of the extension will be implemented by the assigned group member. These extensions can be done in parallel to each other. The testing of each part will begin after the completion of design and implementation.

\item \textbf{FPGA Testing:} (2nd  September 2007 � 15th September 2007)

\item \textbf{System Integration and Testing:} (15th September 2007 -  21st October 2007)

\item \textbf{Final Individual Report:} (milestone: 4th October 2007)

\item \textbf{Final Seminar:} (milestone: week 10 and 11)

\item \textbf{Project Exhibition:} (week 12)
\end{enumerate}

\subsection{Activity Schedule}
%%TODO PUT IN RIGHT SPOT 
\begin{figure}
\centering
\includegraphics[width=\textwidth]{gantt-chart}
\caption{Schedule}
\label{gantt-chart}
\end{figure}


\section{Risk Analysis}

The potential risks in this project are shown in Table~\ref{risks} and some planned approaches are described in this section.

\begin{table}[h]
\begin{tabular}{|l|l|l|l|}
  \hline
  % after \\: \hline or \cline{col1-col2} \cline{col3-col4} ...
  \textbf{Risk} & \textbf{Chance} & \textbf{Impact} & \textbf{Rating} \\
  \hline
  \hline
  1. Project falls behind schedule & moderate & moderate & moderate\\
  \hline
  2. Communication failure & low & high & moderate\\
  \hline
  3. Design bugs & moderate & moderate & moderate\\
  \hline
  4. Faulty hardware parts & low & moderate & low\\
  \hline
  5. Unavailability of resources & moderate & low & low\\
  \hline
  6. Absence of team members & low & low &low\\
  \hline
  7. Change of supervisor & high & low & moderate\\
  \hline
  8. Changing requirements & moderate & low & low\\
  \hline
  9. Fabrication grant is not awarded & low & high & moderate\\
  \hline
\end{tabular}
\caption{Summary of potential risks.}
  \label{risks}
\end{table}

\begin{enumerate}
\item \textbf{Project falls behind schedule:} Some of the project milestones are set by the Harvey Mudd College team. Thisproject will be very heavily loaded in the first semester. It also requires knowledge of microelectronics circuit design, circuit verification, computer architecture etc., therefore the team needs some extra reading and practice on tools that we are using, prior to the commence of the actual work for achieving the target project milestones. Even though this project was commenced before the beginning of the semester, the team member will be required to spend extra time and effort in order to meet the deadlines. Considering the project extensions in the second semester, the low power extensions are lower priority, Rys and Rob can be reallocated if project severely falls behind schedule.

\item \textbf{Communication failure:} This project is conducted in collaboration with the Harvey Mudd College team in California. Regular contact with the other teams in California is important. Design reviews also will be conducted with the aid of video conferencing. Therefore, alternative contact information other than university email, such as telephones and faxes are necessary to prevent unexpected communication failure. Besides, weekly project meetings in Adelaide will include status review, each member can specify their progress and plan in order to prevent digression and allows clarification on actions open to misinterpretation.

\item \textbf{Design bugs:} The cache circuit design was brokedown into blocks and each member are allocated to different blocks to test their functionalities. In order to ensure the reliability of the overall chip design, designing high coverage test benches are required for design bugs detection. Careful simulation checks are very crucial, if test bench flags error, the design will be reviewed to resolve the problem. All blocks should be tested prior to the final testing of the integrated cache circuit. Besides, the Adelaide team will attempt extra system-level verification. Both Adelaide and US teams are performing independent system level tests to enhanced system reliability.

\item \textbf{Faulty hardware parts:} The team is expecting the real chip to be delivered in the second semester to proceed with the project extension on low power design and hardware demonstration. Any faulty hardware part will affect the testing results of the chip. By foreseeing the risk, the FPGA is another alternative if hardware fails. Our backup plan is to use FPGA to demonstrate our project in the end of the year if chip fails to perform as what we expected. 

\item \textbf{Unavailability of resources:} Some software tools may become unavailable in the computer in CATS and EM211. We should submit support requests as soon as problems are noticed. Meanwhile, the team will work on other tasks until the tools become available to reduce the impact. It may be possible to use other tools, e.g. The school of Engineering has multiple Verilog simulators.

\item \textbf{Absence of team members:} If a team member is sick or burdened with other commitments, other team members must continue productive work. The project schedule will be reassessed if necessary, less priority extension might be omitted from project plan.

\item \textbf{Change of supervisor:} The current supervisor, Dr. Branden Phillips will be away in the second semester and his supervising job on this project will be taken over by Dr. Brain W. Ng. It is the responsibility of the team to update the supervisor on the progress of the project, and also be prepared for any adjustment in running project.

\item \textbf{Changing requirements:}The cache circuit will be incorporated into the MIPS microprocessor which is still currently being developed. The team is expecting the change of requirements for the cache circuit throughout the circuit design process in order to optimize the overall chip performance.

\item \textbf{Fabrication grant is not awarded:} There is a possibility that the MIPS microprocessor will not be granted for fabrication in April or later. This will result in a great impact on out project plan for the second semester as we are expecting to get the real chip for testing and further improvement on low power design. Besides, the project demonstration in week 12 required final product to be presented. As a backup plan, we use FPGA for prototyping the MIPS microprocessor design so that it can perform whatever logical function is needed. A FPGA board will be purchased towards the beginning of semester one.
\end{enumerate}



%END MELS SECTION

%JOEL'S SECTION

\section{Budget}

There is an allocation of \$250 per student, making the team budget
\$1000. We have identified one major item to purchase,
that being a Xilinx XCV2P Virtex-II Pro FPGA board. The cost per board is
US\$299, approximately \$370. We need two of these boards, plus a portable housing for the testbed setup to enable easy transportation and demoing of the system.

There may be a requirement to purchace additional hardware components to enable connection of the two FPGA boards together.

\begin{table}[h]
\begin{tabular}{|l|c|c|c|c|}
  \hline
  \textbf{Item} & \textbf{Cost} \\
  \hline
  \hline
  Budget allocation & \$1000 \\
  \hline
  Xilinx FPGA x2 & -\$760 \\
  \hline
  Housing & -\$50 \\
  \hline
  \hline
  \textbf{TOTAL} & \$810 \\
  \hline
\end{tabular}
\caption{Project budget. }
 \label{budget}
\end{table}




%END JOEL's SECTION

% The 99 parameter tells LaTeX there are at most 99 references.
\begin{thebibliography}{99}

% Use this format of references to online material
\bibitem{Green07} C.~A.~Green, \emph{Final Year Honours and Design
  Project Handbook}, 2007, http://www.eleceng.adelaide.edu.au/students/undergrad/courses/4039/material/notes/

\bibitem{various07} Various, \emph{HMC-MIPS Wiki Design Sketches},
2007, http://code.google.com/p/hmc-mips/wiki/DesignSketches

\end{thebibliography}

%\newpage
%\section*{Glossary and Symbols}
%
%\begin{description}
%\item[Discursive Viscosity:] The average number of jargon terms per
%paragraph.
%\item[RNS:] Residue Number System
%\item[TLA:] Three Letter Acronym
%\item[\(\psi\):] Discursive viscosity
%
%\end{description}

\end{document}
