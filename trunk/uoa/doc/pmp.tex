% PROJECT PROPOSAL TEMPLATE
%
% This document was prepared for use with pdflatex.

% Lines beginning with a % are comments.
\documentclass[a4paper,12pt]{article}

% The amsfonts package contains some useful mathematical symbols
\usepackage{amsfonts}

% The graphicx package allows you to import JPG and PDF images
\usepackage[pdftex]{graphicx}

% The enumerate pachage permits fancy enumerated lists
\usepackage{enumerate}

% Setup the page
\topmargin -1.5cm
\textheight 25.0cm
\oddsidemargin -0.0cm
\textwidth 16.5cm
\pagestyle{plain}
\setcounter{page}{0}
\linespread{1.3}

\begin{document}

%*******************************************************************
% Draw the title page
% The only things you are likely to need to edit are marked with
% a **** comment ****
\begin{titlepage}
\vspace{-1.5cm}
\begin{center}
\includegraphics[width=2cm]{ualogo_colour.jpg}
\vspace{1cm}

\textbf{\large THE UNIVERSITY OF ADELAIDE}\\

\textbf{SCHOOL OF ELECTRICAL \& ELECTRONIC ENGINEERING}

{\small ADELAIDE, SOUTH AUSTRALIA, 5005}

\vspace{1.5cm}
\textsc{Implementation Plan}
\vspace{1cm}

% **** Insert your project title here ****
\textbf{\LARGE Project Title}

\vspace{1cm}
% **** Insert the group members' names here ****
{\Large Group Members}

\vspace{4cm}
\textbf{\large ELEC ENG 4039 A/B HONOURS PROJECT}

\vspace{1ex}
\setlength{\linespread}{1}
\textbf{B.E. in Electrical and Electronic Engineering\\
B.E. in Computer Systems Engineering\\
B.E. in Telecommunications Engineering\\}
\end{center}

\vfill
Each student at Level IV of the course in Electrical and
Electronic Engineering, Computer Systems Engineering and
Telecommunications Engineering is required to complete this course.
The course involves approximately 240 hours of project work over the
whole academic year.  Students are assessed on their performance in
the project, a written proposal, this written report, a technical
paper, and two seminar presentations.

\vspace{1cm}
Date submitted:

\vspace{1ex}
Supervisor:

\vspace{1ex}
Signature of Supervisor:
\end{titlepage}
% End of the title page
%*******************************************************************

\newpage
% The \section* command creates a section without a number
\section*{Objectives}
Concisely state the objectives of the project. You should be able to
do this in just a few sentences.

\newpage
\tableofcontents

\newpage
% The \section command creates a numbered section
% Note that there are also \subsection and \subsubsection commands.
\section{Background and Significance}

The first aim of this project is to co-operate with the VLSI class
from Harvey-Mudd College, California, to design and build a MIPS-based
microprocessor.

Microprocessors are electronic devices that contain all the functions
of a CPU on a single integrated circuit. This microprocessor will be
split into a control unit, a coprocessor and a data path. The datapath
contains a fetch, decode, execute and memory stage. The Adelaide team
is designing the 512kB cache for the memory stage. A cache stores
recently used memory data on chip to save slow memory accesses. The
cache is direct mapped, meaning that each datum from memory has a
unique place it can be stored in the cache.

The chip will use a MIPS R2000 instruction set architecture
(ISA). MIPS stands for Microprocessor without Interlocked Pipeline
Stages. The MIPS ISA is a popular RISC microprocessor
architecture. R2000 has 32-bit instructions including various loads,
stores, arithmetic, jumps and branches, shifts, moves and exceptions.

In the production of a microprocessor testing is required on both the
design and the finished hardware.

\section{Project Specifications}

\subsection{Final Deliverables}
\begin{enumerate}[{[D}1{]}]
\item Hardware Presentation of the MIPS-based microprocessor
\item Package of testing tools and report on test results
\item Extension: Report examining low power design alternatives
\end{enumerate}

\subsection{Requirements}
\begin{enumerate}[{[R}1{]}]
\item Hardware demonstration
  \begin{itemize}
  \item Running an interactive program (possibly a Web-server or
    ELIZA).
  \item Robust packaging so that the demonstration is easy to
    move and set up.  
  \item Uses MIPS-based microprocessor.
  \end{itemize}

\item Testing tools
  \begin{itemize}
    \item Software tests of design using either fast SPICE or IRSIM,
      depending on availability.
    \item Documentation for software testing.
    \item Detailed methods of hardware testing.
  \end{itemize}
\item Extension: Report on low power
  \begin{itemize}
  \item Examination of MIPS based microprocessor.
  \item Development of alternative designs to achieve low power.
  \item Examination and evaluation of alternative designs
  \end{itemize}
\end{enumerate}

\subsection{Reporting Requirements}

First Semester
\begin{description}
\item[Week 4 Project Implementation Plan:] A plan detailing what is to
  be accomplished during the project, and how the group intends to do
  it.
\item[Week 5 Proposal Seminar:] Presents and explains the intended
  project including requirements and what will be produced.
\item[Week 8 Critical Design Review:] Evaluate and assess intended
  design(s) for project.
\item[Week 10 Submit Peer Review:] Review another group's project.
\item[Week 12 Project Log Book:] A record of all meetings and time
  spent on the project.
\end{description}

Second Semester
\begin{description}
\item[Week 9 Final Project Report and Technical Paper due:] The final
report covering the completed project and the process of producing it.
\item[Week 10 Final Project Seminars:] A seminar on the completed
project.  
\item[Week 11 Project Exhibition:] Displaying the project to the
public.
\end{description}

%\subsection{Acceptance Criteria}
%
%State how the customer will measure whether the project has been a
%success.

\section{Proposed Approach}

\subsection{System Architecture}

The cache is the focal point of our project. A thorough description of
the memory system being implemented for this project can be found in
\cite{various07}. The MIPS microprocessor was initially designed
by defining an interface and then describing all modules in verilog
using a development tool called ModelSim. Once the verilog system was
completed and verified work began on translating the verilog code into
schematics. The schematics are currently in the process of being
developed and using a VLSI design application called Electric8.04. The
design process will be elaborated in Section 3.2.

Given in Figure~\ref{cacheblock} is the current block diagram of
the cache system. The cache is comprised of:
\begin{enumerate}
\item Cache Controller
\item Cache RAM
\item Data Out
\item Byte Enable
\item Various simple logic components including multiplexers,
  inverters, and OR gates.
\end{enumerate}

\begin{figure}
\centering 
%\includegraphics[width=\textwidth]{cacheblock}
\includegraphics[width=\textwidth]{cacheblock}
\caption{Block diagram of the MIPS cache.}
\label{cacheblock}
\end{figure}

The Adelaide team will be involved in the continued design and
verification of the Cache Controller and the Cache Ram. These modules
are described in further detail in Figure~\ref{cachcontroller} and
Figure~\ref{cacheram} respectively.

\subsection{Cache Controller}

This module comprises the following components: 
\begin{enumerate}
\item Address Tag Data Logic
\item Controller State Logic
\item Inverters, OR gates and flops
\end{enumerate}

\subsection{Cache Ram}

This module contains the following components:
\begin{enumerate}
\item 64-bit Decoder
\item Cache Signal Buffers
\item SRAM Array
\item Bit Line Conditioning
\item Write Driver
\end{enumerate}

\begin{figure}
\centering 
\includegraphics[width=\textwidth]{cachecontroller}
\caption{The MIPS cache controller module.}
\label{cachecontroller}
\end{figure}

\begin{figure}
\centering 
\includegraphics[width=\textwidth]{cacheram}
\caption{The MIPS cache ram module.}
\label{cacheram}
\end{figure}

\subsection{Design Processes}

Figure~\ref{designprocess} describes the methodology that will be
followed to complete the stages of development of the cache
system. The first stage involves starting with the verilog
specification that has already been developed and verified. This
starting point involves learning the verilog hardware description
language and then determining what modules were used and the overall
interface for the memory system. The verilog code was developed using
ModelSim and so familiarity with this development environment is a
requirement.

The next stage of the design flow involves designing schematics from
the verilog code. This will be achieved by using the VLSI design suite
Electric. Once the schematics are designed a verilog deck, containing
the schematic netlist of a module will be extracted and inserted in
place of the original verilog description for that module. This will
enable checking that each schematic that has been designed integrates
with the interface and is logically correct.

Once the schematics have been verified, layouts of the schematics will
commence. A similar testing procedure will be conducted to the
schematics for the layouts in addition to the MOSIS rules, Schematic
vs Layout verification features built-in to Electric.

At present, the intended further testing procedure involves generating
IRSIM command files and spice test files from the verilog
code. Verilog is only designed to conduct logical testing of
circuitry.This is, therefore, an important stage to complete since
IRSIM will be allow evaluation of delays and critical paths within the
circuitry. `Fast Spice' will also be used to check on behaviour of
circuitry that is significant to analogue design.

\begin{figure}
\centering 
\includegraphics[width=\textwidth]{designprocess}
\caption{Flow chart of the design process for the project.}
\label{designprocess}
\end{figure}

%MELS SECTION

\input{pmp-mel}

%END MELS SECTION

\section{Budget}

There is an allocation of $250 per student, making the team budget
$1000. At this stage we have identified only one item to purchase,
that being a Xilinx XCV2P Virtex-II Pro FPGA board. The cost is
US\$299, approximately \$370. 

\begin{table}[h]
\begin{tabular}{|l|c|c|c|c|}
  \hline
  \textbf{Item} & \textbf{Cost} \\
  \hline
  \hline
  Budget allocation & \$1000 \\
  \hline
  Xilinx FPGA & -\$500 \\
  \hline
  \hline
  \textbf{TOTAL} & \$500 \\
  \hline
\end{tabular}
\caption{Project budget. }
 \label{budget}
\end{table}

% The 99 parameter tells LaTeX there are at most 99 references.
\begin{thebibliography}{99}

% Use this format of references to online material
\bibitem{Green07} C.~A.~Green, \emph{Final Year Honours and Design
  Project Handbook}, 2007, http://www.eleceng.adelaide.edu.au/students/undergrad/courses/4039/material/notes/

\bibitem{various07} Various, \emph{HMC-MIPS Wiki Design Sketches},
2007, http://code.google.com/p/hmc-mips/wiki/DesignSketches

\end{thebibliography}

%\newpage
%\section*{Glossary and Symbols}
%
%\begin{description}
%\item[Discursive Viscosity:] The average number of jargon terms per
%paragraph.
%\item[RNS:] Residue Number System
%\item[TLA:] Three Letter Acronym
%\item[\(\psi\):] Discursive viscosity
%
%\end{description}

\end{document}
